\documentclass[20]{jarticle}
\usepackage{ketpic,ketlayer}
\usepackage{amsmath,amssymb}
\usepackage{graphicx}
\usepackage{color}
\usepackage{emath}

\setmargin{20}{20}{20}{20}

\begin{document}

\begin{itemize}
\item タイトル\\
三角関数の値を求める問題演習
\item 目的\\
この教材は,繰り返し多くの問題を解くことによって,三角関数の定義を理解し,三角関数の値を容易に求めることができるようになることを目的とします.
\item 内容\\
弧度法の角 $\theta $が与えられ,それに対して,$\sin \theta ,\ \cos \theta , \tan \theta $の値を求めます.様々な弧度法の角に対する問題が出題されるます.
ヒントもあり,視覚的に理解することもできます.

%三角関数の定義を理解し,三角関数の値を求めることができるようになることは重要です.弧度法の角に対して,その問題演習をします.

\item 使い方\\
次の手順で問題を解きます.
\begin{enumerate}
\setcounter{enumi}{-1}
\item 名前を学習者蘭に入力する.
\item 「出題」 問題が表示される.
\item 解答を解答欄に入力する.答えが定義されないときは,「no」を入力する.\\[1mm]
分からないときはヒントを利用する(分かるときは利用しない).\\[1mm]
\hspace{1zw}「ヒント1」 角が図に表示される.\\[1mm]
\hspace{1zw}「ヒント2」 単位円周上の点の座標が図に表示される.
\item 「決定」 入力した解答が解答欄の右に表示される.
\item 「採点」 採点され正答が表示される.\\[1mm]
\hspace{2zw}次の問題を解くときは,1に戻って「出題」ボタンを押す.\\[1mm]
\hspace{2zw}終了するときは,「記録」ボタンを押す.


\end{enumerate}

\end{itemize}
\end{document}
