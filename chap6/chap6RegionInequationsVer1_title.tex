\documentclass[20]{jarticle}
\usepackage{ketpic,ketlayer}
\usepackage{amsmath,amssymb}
\usepackage{graphicx}
\usepackage{color}
\usepackage{emath}

\setmargin{20}{20}{20}{20}

\begin{document}

\begin{itemize}
\item タイトル\\
連立1次不等式の表す領域

\item 目的\\
1次不等式$y>ax+b$または$y<ax+b$の境界線$y=ax+b$を図示して,その1次不等式の表す領域を求めましょう.
さらに,2つの1次不等式の表す領域の共通部分を指定し,連立1次不等式の表す領域を求めましょう.

\item 内容\\
2つの1次不等式の境界線を図示し,これらの1次不等式の表す領域の共通部分として連立1次不等式の表す領域を求める教材です.

\item 使い方\\
次の手順で操作します.
\begin{enumerate}[(1)]
\item 名前や番号をID欄に入力してください.
\item 「出題」ボタンを押すと,問題が表示されます.
\item 2点A, Bを置き,「\Maru{1}の境界の表示」ボタンを押すと,直線ABが表示され,採点結果と正解の直線が表示されます.
\item 2点A, Bを移動し,「\Maru{2}の境界の表示」ボタンを押すと,直線ABが表示され,採点結果と正解の直線が表示されます.
\item 点Cを連立不等式の表す領域の内部に置き,「採点」ボタンを押すと,点Cの座標および採点結果と正解の領域が表示されます.
\item 最後に「記録」ボタンを押すと,記録が書き込まれます.
さらに問題を解きたい場合は「出題」ボタンを押すと,すべての表示が消えて新しい問題が表示されます.
\end{enumerate}

\end{itemize}
\end{document}
