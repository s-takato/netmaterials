\documentclass[20]{jarticle}
\usepackage{ketpic,ketlayer}
\usepackage{amsmath,amssymb}
\usepackage{graphicx}
\usepackage{color}
\usepackage{emath}

\setmargin{20}{20}{20}{20}

\begin{document}

\begin{itemize}
\item タイトル\\
連立1次不等式の表す領域

\item 内容\\
2つの1次不等式の領域の共通部分を求める教材です.

\item 使い方\\
次の手順で操作します.
\begin{enumerate}[(1)]
\item 名前や番号をID欄に入力してください.
\item 「出題」ボタンを押すと,問題が表示され,不等式\Maru{1}の表す領域の境界が表示されます.
\item 点Aを不等式\Maru{1}の表す領域の内部に置き,「(1)の解答」ボタンを押すと,点Aの座標が表示されます.さらに,「(1)の採点」ボタンを押すと,採点結果と正解の領域が表示されます.
\item 「(2)表示」ボタンを押すと,不等式\Maru{2}の表す領域の境界が表示されます.
\item 点Bを不等式\Maru{2}の表す領域の内部に置き,「(2)の解答」ボタンを押すと,点Bの座標が表示されます.さらに,「(2)の採点」ボタンを押すと,採点結果と正解の領域が表示されます.
\item 「(3)表示」ボタンを押すと,不等式\Maru{1}と不等式\Maru{2}の表す領域の境界が表示されます.
\item 点Cを連立不等式の表す領域の内部に置き,「(3)の解答」ボタンを押すと,点Cの座標が表示されます.さらに,「(3)の採点」ボタンを押すと,採点結果と正解の領域が表示されます.
\item 最後に「記録」ボタンを押すと,記録が書き込まれ,全ての表示が消えます.さらに問題を解きたい場合には,「出題」ボタンを押すと,新たな問題が表示されます.
\end{enumerate}

\end{itemize}
\end{document}
