\documentclass[20]{jarticle}
\usepackage{ketpic,ketlayer}
\usepackage{amsmath,amssymb}
\usepackage{graphicx}
\usepackage{color}
\usepackage{emath}

\setmargin{30}{30}{30}{20}

\begin{document}

\begin{itemize}
\item タイトル\\
与えられた数列の第5項までを求める問題
\item 内容\\
等差数列,等比数列,漸化式で与えられた数列がランダムに表示され,
その形式によって数列の項が次々に作成されていくこと確認する.

\item 使い方\\
次の手順で問題を解きます.
\begin{enumerate}
\setcounter{enumi}{-1}
\item 名前または番号ををID欄に入力する.\vspace{1mm}\\
「確認」ボタンを押すと問題が開始される..\vspace{1mm}
\item 「出題」ボタンを押すと新たな問題が表示される.\vspace{1mm}\\
ただし,解答の途中で「出題」ボタンを押すと途中の結果は保存されない.\vspace{1mm}
\item 解答を解答欄に入力する.\vspace{1mm}
\item 「決定」 ボタンを押すと「正解」「不正解」が表示される.\\[1mm]
ただし,すべての解答欄に数値が代入されている必要がある.\\[1mm]
正解のときは,一般項が表示される.一般項の形を確認すること\vspace{1mm}
\item 次の問題を解くときは,1に戻って「出題」ボタンを押す.\vspace{1mm}
\item 終了するときは,「記録」ボタンを押す.\\[1mm]
途中で「記録」ボタンを押すとそれまでの結果が表示される.\\[1mm]
「出題」ボタンを押すと記録欄はクリアされるがそれまでの結果は保存されている.
\end{enumerate}

\end{itemize}
\end{document}
