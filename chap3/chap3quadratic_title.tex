\documentclass[20]{jarticle}
\usepackage{ketpic,ketlayer}
\usepackage{amsmath,amssymb}
\usepackage{graphicx}
\usepackage{color}
\usepackage{emath}

\setmargin{20}{20}{20}{20}

\begin{document}

\begin{itemize}
\item タイトル\\
%三角関数の値を求める練習問題
2次関数の頂点とグラフの概形を求める練習問題

\item 目的\\
2次関数$y=ax^2+bx+c$を標準形$y=a(x-p)^2+q$に直せないと,頂点もグラフも求めることができません.
標準形に変形できるようにしましょう.

\item 内容\\
2次関数$y=ax^2+bx+c$の頂点とグラフの概形を求めます.

\item 使い方\\
次の手順で問題を解きます.
\begin{enumerate}
\setcounter{enumi}{-1}
\item 名前を学習者蘭に入力する.
\item 「出題」 ボタンで問題が表示される.
\item 標準形を求め,点Tを問題のグラフの頂点に,他の点P1~P4を問題のグラフの形になるように配置する.
Tの座標は頂点の座標として画面に表示されている.グラフは頂点を含めた部分的なものでよい.
\item 「採点」 ボタンで頂点とグラフの解答が採点され,標準形と正解のグラフが表示される.
\item 次の問題を出すときは,1に戻って「出題」ボタンを押す.
\item 学習記録をするときは,「記録」ボタンを押す.\\[1mm]
\hspace{2zw}表示された記録にカーソルを置き,\fbox{Ctrl}+\fbox{a\vphantom{l}} と \fbox{Ctrl}+\fbox{c\vphantom{l}} でコピーできる.


\end{enumerate}

\end{itemize}
\end{document}
