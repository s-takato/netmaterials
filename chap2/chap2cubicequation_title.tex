\documentclass[20]{jarticle}
\usepackage{ketpic,ketlayer}
\usepackage{amsmath,amssymb}
\usepackage{graphicx}
\usepackage{color}
\usepackage{emath}

\setmargin{20}{20}{20}{20}

\begin{document}

\begin{itemize}
\item タイトル\\
三次方程式を解く
\item 内容\\
三次方程式を解くための練習問題です.

\item 使い方\\
次の手順で問題を解きます.
\begin{enumerate}
\setcounter{enumi}{-1}
\item 名前を学習者欄に入力する.
\item 「出題」 問題が表示される.
\item 解答を「解は$x=$」の右にある解答欄に入力する.例えば,$-2,1,3$など,カンマで区切って入力する.重解の場合も,$-2,1,1$のように入力する.\\[1mm]
分からないときはヒントを利用する(分かるときは利用しない).\\[1mm]
\hspace{1zw}「ヒント」 解の中で最も小さい値がヒントとして表示される\\[1mm]
\item 「採点」 採点され正答が表示される.\\[1mm]
\hspace{2zw}次の問題を解くときは,1に戻って「出題」ボタンを押す.\\[1mm]
\hspace{2zw}終了するときは,「記録」ボタンを押す.


\end{enumerate}

\end{itemize}
\end{document}
