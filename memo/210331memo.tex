\documentclass[10pt]{ujarticle}

\usepackage{ketpic,ketlayer}
\usepackage{graphicx}
\usepackage[dvipdfmx]{pict2e}
\usepackage{ketlayermorewith2e}
\usepackage{enumerate}
\usepackage{url}

\setmargin{15}{15}{10}{10}

\pagestyle{empty}

\begin{document}

\begin{flushright}
2021.03.31高遠  
\end{flushright}

\begin{center}
{\large ネット教材基礎数学打ち合わせメモ}
\end{center}

\begin{enumerate}[\bf 1.]
\item 全体
\begin{itemize}
\item 全体のフォーマットを統一\\
 学籍番号(名前),日時,学習履歴\\

\item 画面の大きさもほぼ同じにする\\

\item ドリル型の流れ\\
 学生情報=>問題提出(複数可)=>解答(=>ヒント)=>採点=>履歴=>提出\\
\item 対話型の流れ\\
 (学生情報=>)実行(=>記述で解答・感想=>提出)\\

\item 日時や解答時間作成の関数を作成\\

 \item 次回までに説明\TeX ファイルを作成\\

\item 反復練習教材\\

\end{itemize}

\item 各教材の検討
\begin{itemize}
\item[]複素数の和(野澤)\\

\item[]3次方程式(岡崎)\\

\item[]2次関数(赤池)\\

\item[]指数対数(濱口)\\

\item[]三角関数(西浦)\\

\item[]不等式と領域(山下)\\

\item[]数列・漸化式(前田)\\

\item[]パスカルの三角形(前田)\\

\item[]指数対数(高遠)\\

\item[]2次曲線(高遠)\\

\end{itemize}

\item 日程

 \Ltab{8zw}{締切}2021.06%\vspace{1zw}

 \Ltab{8zw}{ネット公開}2021.06\vspace{1zw}

\end{enumerate}

\end{document}
