\documentclass[20]{jarticle}
\usepackage{ketpic,ketlayer}
\usepackage{amsmath,amssymb}
\usepackage{graphicx}
\usepackage{color}
\usepackage{emath}

\setmargin{20}{20}{20}{20}

\begin{document}

\begin{itemize}
\item タイトル\\
複素数の和と複素数平面
\item 内容\\
複素数の足し算を複素数平面上で考える
ための教材です
\item 使い方\\
次の手順で操作します.
\begin{enumerate}[(1)]
\item 名前や番号をID欄に入力し,「確定」ボタンを押す.
\item 複素数平面上の赤い点をリックして$\alpha+\beta$の位置までドラッグした後,「決定」ボタンを押す.
\item 赤い点の位置が正しければ「正解」,正しくなければ「不正解」の文字が表示される.
\item 「問題」のボタンを押し,$\alpha$と$\beta$の値を変更し, 上の操作を適宜繰り返す.
\item 操作を繰り返してわかったことを「問題を解いてわかったこと」の下の欄に記述し,「記録」ボタンを押す.

\end{enumerate}

\end{itemize}
\end{document}
