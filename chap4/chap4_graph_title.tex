\documentclass[10pt]{jarticle}

\usepackage{ketpic,ketlayer}
\usepackage{graphicx}
\usepackage[dvipdfmx]{pict2e}
\usepackage{ketlayermorewith2e}
\usepackage{enumerate}
\usepackage{url}

\setmargin{15}{15}{10}{10}

\pagestyle{empty}

\begin{document}

\begin{itemize}
\item タイトル\\
指数関数と対数関数のグラフ

\item 目的\\
指数関数と対数関数のグラフについて,底の値による形状の違いや,逆関数としての性質である直線$y=x$に関する対称性を理解する.

\item 内容\\
底の値を変えると,指数関数と対数関数のグラフがどのようになるか確認します.\\
$x$の値を変化させると,指数関数の値がどのようになるか,グラフ上のPの座標として考えます.\\
$x$の値と対数関数の値がどのように対応しているか,また,指数関数とはどのような関係にあるか,グラフ上のQの座標を利用して考えます.

\item 使い方
\begin{enumerate}[(1)]
\item
``底の値''のスライダーの変化にあわせて指数関数と対数関数のグラフが変化します.底の値が大きいときや小さいとき,それぞれのグラフの形状がどのようになるか確認します.
\item
点Pの$x$座標のスライダーを変化させると,指数関数の値がグラフ上のPの$y$座標として表されます.
大きな$x$の値に対する点はすぐに見えなくなりますが,実際にはどのくらいの大きさの関数の値になるかが数値で表されます.
\item
点Qの$x$座標を入力すると,対数関数の値がグラフ上のQの$y$座標として表されます.
``とても大きな$x$の値''を入れると関数の値がどれくらいになるのか,指数関数との逆関数の関係を考えながら入力することができます.
\end{enumerate}
\end{itemize}


\end{document}
